\chapter{Supplementary Information to \texorpdfstring{\Cref{sec:chapter-basiss-applications}}{}}


\section*{Declaration}
This supplementary chapter presents tables and figures that complement \cref{sec:chapter-basiss-applications}. All content derives from the supplementary materials of \textcite{Lomakin2022-ks}, although with minor stylistic modifications.

{
\footnotesize
\begin{longtable}{lp{3cm}p{3cm}}
    \tabcap{case-descsription}{Patient clinical description}{Clinical information for the studied breast cancer cases} \\
    \toprule
    Parameter & P1 case & P2 case \\
    \midrule
    Manuscript ID & P1 & P2 \\
    Case & PD9694 & PD14780 \\
    Sex & Female & Female \\
    Age at Diagnosis (Years) & 37 & 66 \\
    Children & Unknown & Yes \\
    Maximum Tumour Size (mm) & 26 & Not Recorded \\
    Axillary Nodes & Micromets (1LN) & 3+ \\
    Metastasis Status at Diagnosis & M0 & M0 \\
    Histology & Invasive Carcinoma: No Special Type & Invasive Carcinoma: No Special Type \\
    In Situ Component & Extensive Intermixed DCIS & No \\
    Multifocal (Number of Invasive Foci) & Yes (2) & Yes (3) \\
    Grade Primary & 2 & Unknown \\
    ER Status Primary & Positive & Negative \\
    PR Status Primary & Positive & Negative \\
    HER2 Primary (IHC) & Negative & Negative \\
    \bottomrule
\end{longtable}
}
\clearpage
{
\begin{sidewaystable}
    \footnotesize
    \begin{longtable}{p{1.6cm}p{1.6cm}p{1.6cm}p{1.6cm}p{1.6cm}p{1.6cm}p{1.6cm}p{1.6cm}p{1.6cm}}
    \tabcap{sample-descsription}{Sample clinical description}{Clinical data related to the studied breast cancer samples} \\
    \toprule
    Variable & P2-TN1 & P2-TN2 & P2-LN & P1-ER1 & P1-ER2 & P1-D1 & P1-D2 & P1-D3 \\
    \midrule
    Manuscript ID & P2 & P2 & P2 & P1 & P1 & P1 & P1 & P1 \\
    Case & PD14780 & PD14780 & PD14780 & PD9694 & PD9694 & PD9694 & PD9694 & PD9694 \\
    Sub-sample & a & d & e & a & c & d & l & m \\
    Type & Primary invasive tumour & Primary invasive tumour & Lymph node with metastasis & Primary invasive tumour & Primary invasive tumour & DCIS & DCIS & DCIS \\
    Path Review (Frozen H\&E Summary) & Admixed carcinoma, stroma and lymphocytes scattered throughout but in differing proportions. & Admixed carcinoma and stroma, lymphoid island and necrotic areas & Lymph node replaced by carcinoma, cannot see capsule. Dense islands of tumour, may be in sinuses or discrete deposits. & Mainly invasive tumour (NST) with areas of DCIS & Mainly invasive tumour (NST) with areas of DCIS & DCIS and cancerisation of lobules (cribiform features, necrosis, combination of intermediate grade and intermediate to high grade) & DCIS and cancerisation of lobules (cribiform features, necrosis, combination of intermediate grade and intermediate to high grade) & DCIS and cancerisation of lobules, several normal lobules and a duct transforming from DCIS to normal along length \\
    Tissue Size (mm\textsuperscript{2}) & 38 & 68.5 & 44 & 62.36 & 51.3 & 132 & 34 & 58 \\
    Nuclei Count & 224826 & 261305 & 394895 & 319925 & 298682 & 642256 & 173508 & 104523 \\
    \bottomrule
    \end{longtable}
\end{sidewaystable}
}
\clearpage
\appendixfigurefloat{appendix-applications-PBC-P1.pdf}{appendix-applications-PBC-P1}
    {Phenotype characterisation of histo-genomic states in sample P1 PBCs}
    {\textbf{a}, Broadly annotated H\&E tissue sections of the P1-ER1 and P1-ER2 primary breast cancers. \textbf{b}, Microregions selected for detailed analysis overlaid on \ac{BaSISS} maps (regions relate to heatmaps in Fig. 3a; numbers relate to histological annotations in Supplementary Table 2). c, Comparison of the \acf{CCF} of 9 regions of P1-ER1/P1-ER2 determined through both \ac{BaSISS} (top) and \acf{LCM} \acf{WGS} (bottom). \textbf{d}, Snapshots of \acf{IHC} staining in serial fresh frozen tissue cryosections from P1-ER2. Selected regions with confirmed clone compositions (by \ac{LCM}-\ac{WGS}) are presented. \protein{SMMHC}/\protein{P63} antibody stains myoepithelial cells red, \protein{PTEN} protein and the progesterone receptor (\protein{PR}) stain brown. \% reports proportion of positive nuclei stained, n reports number of nuclei in region assessed by QuPath digital software. Row 1–3 scale bars = 250 $\mu$m. Row 4 scale bar = 50 $\mu$m. \textbf{e}, Violin plots depict clone specific \protein{Ki67} \ac{IHC} staining rate posterior density of the generalised linear mixed model (glmm) with region specific random effect. Significant comparisons were controlled for \ac{FDR} using the \ac{BH} procedure. Analysis was limited to the 11 regions with confirmed clone compositions by WGS due to variation between \ac{IHC} and \ac{BaSISS} sections in z-stack morphology (relates to the bottom panel on \cref{fig:basiss-data-raw}). \textbf{f}, Violin plots depict clone specific gene expression contribution posterior density of the glmm with region specific random effect. A total of 36 regions of P1-ER2 with a dominant clone fraction > 0.7 were analysed. Significant comparisons were controlled for \ac{FDR} using the \ac{BH} procedure. \ac{DCIS} - Ductal carcinoma in situ.}

\appendixfigurefloat{appendix-applications-PBC-P2.pdf}{appendix-applications-PBC-P2}
    {Ecosystem characterisation in P2-TN1}
    {\textbf{a}, H\&E stained sections of the two primary breast cancers from case P2. \textbf{b}, Microregions selected for detailed analysis overlaid on BaSISS maps (regions relate to heatmaps in top \cref{fig:applications-maps-PBC-P2}; numbers relate to histological annotations in Supplementary Table 2). Microregions were not defined for P2-TN2 as a single clone was targeted and detected. \textbf{c}, Cell type contribution posterior density of the generalised linear mixed models (glmm) model with region specific random effect. Significant comparisons were controlled for \ac{FDR} using the \ac{BH} procedure. 19 clone territories (with dominant clone fraction > 0.1) were analysed. Fibroblasts and perivascular-like cells (PVL) could not be differentiated within this experiment and are reported as `fibroblasts'. \textbf{d}, Volcano plot of epithelial expression of the 91 oncology \ac{ISS} panel genes in TN1 invasive regions. Significance was adjusted for multiple testing using BH procedure, only genes with \ac{FDR} < 0.1 and fold change > 1.5 in both ways are coloured/labelled. \ac{DCIS} = Ductal carcinoma in situ.}

\appendixfigurefloat{appendix-applications-DCIS.pdf}{appendix-applications-DCIS}
    {DCIS clone specific histologies}{\textbf{a}, \ac{BaSISS} clone map of P1-D3, a sample that contains Ductal carcinoma in Situ (DCIS), stroma and normal glandular regions. The most prevalent genetic clone colour is projected as a coloured field on DAPI images (reported if cancer cell fraction > 25\% and inferred local cell density > 300 cells/mm2). Scale bar = 5 mm. Inlaid, H\&E stained image (from a serial tissue section) details the histological transition from normal to DCIS morphology, consistent with the clone field transition in the \ac{BaSISS} map (scale bar = 1 mm). \textbf{b}, Heatmap of \ac{CCF} derived from \ac{LCM}-\ac{WGS} of six regions of P1-D1/P1-D2 with cartoon of predicted clone composition indicating inference of monoclonal and polyclonal growth patterns. \textbf{c}, Example of a clone interface within a single sub-lobular space in P1-D1. Clone fields (top left); spatial \ac{BaSISS} mutation signals (top right); characteristic histological features on H\&E (bottom left) with zoom image of clone interface (scale bar = 100 $\mu$m) (bottom right). \textbf{d}, Histological, genetic and transcriptional features of three lobules (identified on the clone map of P1-D2; left, scale bar = 5 mm) are shown: H\&E staining (top) scale bar = 1 mm; \ac{BaSISS} clone fields projected on DAPI with frequency plots of the local, mean cancer (coloured areas) and non-cancer (white) corresponding to horizontal dashed line (middle); and \ac{ISS} gene expression signals reporting \gene{CCND1} and \gene{KRT8} that exhibit clone specific spatial patterns. \textbf{e}, Clone maps of P1-D1/P1-D2 (as presented in \cref{fig:applications-maps-DCIS}) but microregions are coloured according to histological grade. \textbf{f}, Histopathological annotations for each microregion presented alongside the same clone composition heatmap as shown in \cref{fig:applications-maps-DCIS}.}

\appendixfigurefloat{appendix-applications-DCIS-expression.pdf}
    {appendix-applications-DCIS-expression}
    {Distinct transcriptional profiles of two \acs{DCIS} clones}{\textbf{a}, Volcano plot of epithelial expression of the 91 oncology \ac{ISS} panel genes in P1-D2. Significance was adjusted for multiple testing using BH procedure, only genes with \ac{FDR} < 0.1 and fold change > 1.5 in both ways are coloured/labelled. The coloured genes are included in the by region plot in \textbf{b}. \textbf{b}, Heatmap of gene expression data within each of the 41 sampled regions in P1-D2, presented alongside the relevant clone composition regions (top) as per \cref{fig:applications-maps-DCIS}. \ac{ISS} counts in each regions are transformed by applying Poisson cdf with $\lambda$ = mean (P1-green expression, P1-orange expression) $\times$ nuclei count in each region, thus divergence from 0.5 reflects deviation from the global mean expression. Only genes with \ac{FDR} < 0.1 are presented and ordered by the confidence of differential.}

\appendixfigurefloat{appendix-applications-LN.pdf}{appendix-applications-LN}
    {Highly recurrent clone specific ecosystems in a metastatic lymph node}{\textbf{a}, P2-LN1 sample (left) DAPI image with \ac{BaSISS} subclone fields (as shown in \cref{fig:applications-maps-LN}) and coloured squares mark regions depicted in \textbf{b,c,d}; (middle) pan-cytokeratin \ac{IHC} stained (epithelial cells appear brown); (right) CD45 antibody (immune cells appear brown) with ISS immune panel derived cell types projected as coloured dots. \textbf{b}—\textbf{d}, Snapshots of example regions dominated by P2-blue or P2-orange clones, as indicated in \textbf{a}. In each case signals (dots) from selected targets in BaSISS \textbf{b}, \ac{ISS} oncology \textbf{c} or \ac{ISS} immune panels \textbf{d} are presented overlaid on sections stained by \ac{IHC} following the \ac{BaSISS}/\ac{ISS} experiment. In the bottom row of \textbf{c} and top row of \textbf{d} inferred epithelial and immune cell types are presented. In top rows of \textbf{c} and \textbf{d}, 80\% transparency is applied to the underlying \ac{IHC} image to aid visualisation of overlaid dots.}

\appendixfigurefloat{appendix-applications-LN-hypoxia.pdf}{appendix-applications-LN-hypoxia}
    {Hypoxic signature in a metastatic lymph node}{\textbf{a}, Spatial patterns of three hypoxia related genes are projected on the entire P2-LN1 tissue section. \textbf{b}, Spatial patterns of \gene{PDGFRB}, \gene{CD34}, \gene{CD68} and hypoxia related \ac{ISS} signals overlaid on \protein{HER2} (left) and \protein{CD45} \ac{IHC} stained sections(right) correspond to region of white square on top left clone field image in \textbf{a}.}