\chapter{Supplementary Information to \texorpdfstring{\Cref{sec:chapter-basiss-model}}{}}


\section*{Declaration}
This supplementary chapter presents tables that complement \cref{sec:chapter-basiss-model}. All content derives from the supplementary materials of \textcite{Lomakin2022-ks}, although with minor stylistic modifications.

{
\footnotesize
\begin{longtable}{lccccccc}
    \tabcap{genotype-matrix-P1}{Genotype matrix for P1}{The genotype matrix for allele copy numbers serves a critical role in inferring the clone map. This matrix contains numerical values that signify the presumed copy number state of allelic variants in each clone. It is derived as outlined in \cref{sec:protocol-suppl-evo-history}} \\
    \toprule
    Locus & P1-grey & P1-green & P1-purple & P1-blue & P1-red & P1-orange & wt \\
    \midrule
    \gene{SF3B1}, mut/wt   & 1 / 2 & 1 / 2 & 1 / 2 & 1 / 2 & 1 / 2 & 1 / 2 & 0 / 2 \\
    \gene{STUB1}, mut/wt   & 1 / 2 & 1 / 2 & 1 / 2 & 1 / 2 & 1 / 2 & 1 / 2 & 0 / 2 \\
    \gene{CREBBP}, mut/wt  & 1 / 2 & 1 / 2 & 1 / 2 & 1 / 2 & 1 / 2 & 1 / 2 & 0 / 2 \\
    \gene{ARHGEF28}, mut/wt & 1 / 2 & 1 / 2 & 1 / 2 & 1 / 2 & 1 / 2 & 1 / 2 & 0 / 2 \\
    \gene{KIAA0652}, mut/wt & 0 / 2 & 1 / 1 & 0 / 2 & 0 / 2 & 0 / 2 & 0 / 2 & 0 / 2 \\
    \gene{OXSM}, mut/wt & 1 / 2 & 1 / 1 & 1 / 1 & 1 / 1 & 1 / 2 & 1 / 2 & 0 / 2 \\
    \gene{CKAP5}, mut/wt & 0 / 2 & 1 / 1 & 1 / 1 & 1 / 1 & 0 / 2 & 0 / 2 & 0 / 2 \\
    \gene{DENND1A}, mut/wt & 0 / 2 & 0 / 2 & 0 / 2 & 0 / 2 & 1 / 0 & 1 / 0 & 0 / 2 \\
    \gene{NOB1}, mut/wt & 1 / 2 & 1 / 2 & 1 / 1 & 1 / 2 & 1 / 0 & 1 / 0 & 0 / 2 \\
    \gene{RELA}, mut/wt & 0 / 0 & 0 / 0 & 0 / 0 & 0 / 0 & 1 / 0 & 1 / 0 & 0 / 0 \\
    \gene{PLXNA2}, mut/wt & 1 / 2 & 1 / 1 & 1 / 1 & 1 / 2 & 1 / 2 & 1 / 2 & 0 / 2 \\
    \gene{PQLC2}, mut/wt & 0 / 2 & 1 / 1 & 1 / 1 & 1 / 2 & 0 / 2 & 0 / 2 & 0 / 2 \\
    \gene{LRP1B}, mut/wt & 0 / 2 & 0 / 2 & 1 / 1 & 0 / 2 & 0 / 2 & 0 / 2 & 0 / 2 \\
    \gene{PTEN1}, mut/wt & 0 / 2 & 0 / 2 & 0 / 2 & 0 / 2 & 1 / 0 & 0 / 2 & 0 / 2 \\
    \gene{PTEN2}, mut/wt & 0 / 2 & 0 / 2 & 1 / 0 & 0 / 2 & 0 / 2 & 0 / 2 & 0 / 2 \\
    \gene{TMEM8A}, mut/wt & 0 / 2 & 0 / 2 & 1 / 1 & 0 / 2 & 0 / 2 & 0 / 2 & 0 / 2 \\
    \gene{DSEL}, mut/wt & 0 / 3 & 0 / 3 & 1 / 3 & 0 / 3 & 0 / 4 & 0 / 4 & 0 / 2 \\
    \gene{FGFR1exp} & 3 & 3 & 4 & 3 & 16 & 12 & 2 \\
    \gene{KIF14}, mut/wt & 0 / 1 & 0 / 1 & 1 / 1 & 0 / 1 & 0 / 2 & 0 / 2 & 0 / 2 \\
    \gene{AMZ1}, mut/wt & 0 / 2 & 0 / 2 & 1 / 1 & 0 / 2 & 0 / 2 & 0 / 2 & 0 / 2 \\
    \gene{KCNT1}, mut/wt & 1 / 2 & 1 / 2 & 1 / 1 & 1 / 2 & 1 / 2 & 1 / 2 & 0 / 2 \\
    \gene{FZD4}, mut/wt & 0 / 1 & 0 / 1 & 1 / 1 & 0 / 1 & 0 / 2 & 0 / 2 & 0 / 2 \\
    \gene{AP3B22}, mut/wt & 1 / 2 & 1 / 2 & 1 / 1 & 1 / 2 & 1 / 2 & 1 / 2 & 0 / 2 \\
    \gene{EMILIN2}, mut/wt & 0 / 2 & 0 / 2 & 1 / 1 & 0 / 2 & 0 / 2 & 0 / 2 & 0 / 2 \\
    \gene{CCDC105}, mut/wt & 1 / 2 & 1 / 2 & 1 / 1 & 1 / 2 & 1 / 2 & 1 / 2 & 0 / 2 \\
    \gene{ZNF468}, mut/wt  & 1 / 2 & 1 / 2 & 1 / 1 & 1 / 2 & 1 / 2 & 1 / 2 & 0 / 2 \\    
    \end{longtable}
}

\begin{table}[ht]
    \tabcap{genotype-matrix-P2}{Genotype matrix for P2}{The genotype matrix for allele copy numbers serves a critical role in inferring the clone map. This matrix contains numerical values that signify the presumed copy number state of allelic variants in each clone. It is derived as outlined in \cref{sec:protocol-suppl-evo-history}}
    \begin{minipage}{\textwidth}
        \footnotesize
        \centering
        \begin{tabularx}{\textwidth}{l *{5}{>{\centering\arraybackslash}X}}
            \toprule
            Locus & P2-blue & P2-green & P2-orange & P2-purple & wt \\
            \midrule
            \gene{TP53}, mut/wt    & 2 / 0 & 2 / 0 & 2 / 0 & 2 / 0 & 0 / 2 \\
            \gene{COX19}, mut/wt   & 1 / 2 & 2 / 2 & 2 / 2 & 2 / 3 & 0 / 2 \\
            \gene{PLXNA1}, mut/wt  & 1 / 4 & 1 / 4 & 1 / 4 & 1 / 3 & 0 / 2 \\
            \gene{RPL37A}, mut/wt  & 0 / 4 & 0 / 4 & 0 / 4 & 1 / 4 & 0 / 2 \\
            \gene{ERBB2}, mut/wt   & 0 / 2 & 1 / 5 & 30 / 150 & 0 / 2 & 0 / 2 \\
            \gene{ERBB2exp}        & 2 & 52 & 52 & 2 & 2 \\
            \gene{CACNB1ex9-ex10wt} & 3 & 4 & 80 & 3 & 2 \\
            \gene{CACNB1intr10-ex10} & 0 & 0 & 20 & 0 & 0 \\
            \bottomrule
        \end{tabularx}
    \end{minipage}
\end{table}
