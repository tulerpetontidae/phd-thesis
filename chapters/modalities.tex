\chapter{Modalities}
\label{sec:chapter-basiss-multimodal}

\section*{Declaration}

\section{Multimodal Analsis}

bla bla bla, usually could be automated but ofte recuires tailored approach



\section{Reuslts}

Knowing the spatial distribution of clonal densities with the BaSISS model described in \cref{sec:chapter-basiss-model}, one may characterise the clones phenotypically based on additional spatially matched data. These include:

\begin{itemize}
    \item histopathological phenotype
    \item \ac{ISS} expression signals assigned to nuclei
    \item cell type annotations
    \item IHC staining
\end{itemize}

These modalities of spatially resolved information are located on several consecutive slides of tissue which could be distorted from one layer to another. To integrate information among all the modalities, we annotated structurally similar areas of breast tissue on each slide which should be in physical proximity on the z-stack. 

\subsection{\acl{ISS} data}

We used In Situ Sequencing (\ac{ISS}) technology to characterise the transcriptional phenotype of tissue samples. The \ac{ISS} dataset included two panels: a previously published 91-gene panel targeting oncology-related markers and a novel 62-gene immune panel designed for immune cell profiling. I assessed the immune panel's performance by correlating its expression levels with paired bulk RNA-seq data in the Triple Negative Breast Cancer (TNBC) case outlined in \cref{sec:applications-cohort-introduction}. The choice of this specific case stemmed from its elevated levels of immune cells, the panel's designed target. Once normalised for technological differences, data exhibited a significant correlation (R = 0.69–0.78, Pearson's) as shown in \pcref{fig:iss-immune-validation}.

Transcript detection, particularly differential expression near clones, often provides functional insights if a gene has a clear association with a specific cell type. For instance, the presence of the \gene{MS4A1} gene strongly indicates the existence of B-cells. However, many genes lack a clear association with a cell type, complicating interpretation. The detection of \gene{ACTA2}, for example, could signify the presence of either smooth muscle cells, myofibroblasts, or basal epithelial cells. Additionally, functional interpretation is confounded by the cell signal's origin; epithelial-mesenchymal transition (EMT) markers in epithelial cells signify transition, whereas the same markers in stromal cells are standard. Therefore, accurate interpretation of \ac{ISS} data requires knowledge of the originating cell type.

While the \ac{ISS} panel aimed to identify specific cell types and was partially based on the OncotypeDX panel, the gene selection process did not consult single-cell atlases. Instead, the chosen genes were primarily informed by the existing literature, which often focuses on protein characteristics unmeasurable by \ac{ISS}. Moreover, the data was generated using an older version of the \ac{ISS} technology based on sequencing by ligation, resulting in a low signal count per cell (fewer than 3 signals per cell when assigned).

Given the data's sparsity, the use of sophisticated methods capable of resolving niche cell types based on expression signatures was impractical. Consequently, I opted for an alternative approach: employing \ac{scRNAseq} data to first identify cell markers that unequivocally define cell types, and subsequently using these markers for tissue profiling.

\subsection{Hierarchical logisitc regression for cell marker discovery}

The objective is to identify easily usable cell type markers for classifying individual cells within a tissue. Ideally, these markers should be unique to a specific cell type or, alternatively, a unique combination of markers should denote a specific cell type. Identifying unique markers is challenging because most genes are involved in cellular programmes that multiple cell types share, particularly when the classification is granular. Current methods for discovering cell type markers rely on statistical tests, heuristic approaches for dimensionality reduction, or machine learning classification models coupled with feature importance analysis. These methods generally fail to explicitly incorporate the concept of combinatorial encoding of cell types.

One natural approach to improving cell type identification exploits the hierarchical organisation of cell types. In this model, we assume that broad cell markers, such as those for epithelial or immune cells, are obligatorily expressed across their respective subtypes. Nonetheless, we permit subtypes to display non-unique markers that might be shared among multiple groups as soon as this group belongs to a different broad type. The key requirement is that the broad cell type remains identifiable.

Consider, for example, the immune-cycling and epithelial-cycling cell subtypes. Both would express markers indicative of cell cycling. Traditional marker discovery methods often misidentify these as being too general and discard as both epithelila and immune groups types express them. However, in combination with the broad marker for immune cells, we can unambiguously categorise the cell as an immune-cycling subtype.

To formalise this strategy, one could employ a marker selection model like logistic regression restricting the model to groups of cells that belong to the same broad cell type. Then, iterate repeat this approach across multiple layers of hierarchy. However, this isolated approach hinders information sharing across hierarchical levels and could yield suboptimal results. To address this limitation, I have restructured the standard logistic regression model to operate within the cell hierarchy framework, aiming to identify cell type markers at each hierarchical level.

\subsubsection*{Multi-class logistic regression}

Consider a Bayesian formulation of the logistic regression model for predicting cell types. Let $y_i$ represent the cell type label for cell $i$, which can assume one of $K$ possible classes, $\{1, 2, \ldots, K\}$. Define $\mathbf{x}_i$ as the $g$-dimensional vector of gene expression levels for cell $i$. The likelihood function for this model is given by:

\begin{equation}
    \label{eq:logistic-regression-likelihood}
    p(y_i = k | \mathbf{x}_i, \mathbf{w}) = \frac{\exp(\mathbf{w}_{k}^T \mathbf{x}_i)}{\sum_{k=1}^{K} \exp(\mathbf{w}_{k}^T \mathbf{x}_i)}
\end{equation}

Here, $\mathbf{w}_k$ is the weight vector corresponding to class $k$. To promote sparsity, we employ a zero-centred Laplace prior for the weights, characterised by scale parameter $\lambda$:

\begin{equation}
    \label{eq:logistic-regression-prior}
    p(\mathbf{w}_k) \sim \text{Laplace}(0, \lambda)
\end{equation}

We fit this model to an annotated \ac{scRNAseq} dataset using the \ac{VI} method, as detailed in \cref{sec:bayesian-intro}. This provides an approximation of the posterior distribution $p(\mathbf{w}_k | \mathbf{X}, \mathbf{y})$. The magnitude of each $w_{k,g}$ thereby affords a straightforward interpretation of gene $g$'s importance in classifying cell type $k$.

\subsubsection*{Graph structure of annotated single-cell data}

The graphical structure of single-cell data annotation hierarchies can be mathematically represented as a tree $T = (V, E)$, where $V$ is the set of vertices and $E$ is the set of edges. Each vertex $v \in V$ represents a particular cell type annotation, and each edge $(u, v) \in E$ represents a subclass relationship between cell types $u$ and $v$. 

Let $r$ be the root node, representing the most general cell type encompassing all cells. Nodes directly connected to $r$ represent broad cell types; for example these could be ``Epithelial cells'', ``Immune cells'' or ``Stromal cells''. The set of all nodes connected to the root is denoted as $L_1$ and represents the first layer of the hierarchy. 

Mathematically, a hierarchical layer $L_i$ consists of nodes that are equidistant from the root node $r$. That is, for any node $v$ in $L_i$, the shortest path from $r$ to $v$ has length $i$:

\begin{equation}
    L_i = \{v \in V : d(r, v) = i\}
\end{equation}

where $d(r,v)$ denotes the shortest path from $r$ to $v$ in tree $T$. 

For convinience, let's also define a function $f_{\text{p}}$ that retreives the parent $u$ of the child node $u$:

\begin{equation}
    f_{\text{p}}(v) = u, \text{ where } (u, v) \in E
\end{equation}

The $f_{\text{s}}$ function returns the set of children of node $v$, that lie in the same layer as $v$ node, and belong to the same parent node $u$:

\begin{equation}
    f_{\text{s}}(v) = \{w \in V : f_{\text{p}}(w) = f_{\text{p}}(v)\}
\end{equation}

Taking an element in $L_3$, and iteratively applying $f_{\text{p}}$ until we reach $L_1$  would return it's hierachy ``CD8+ T cells'' $\rightarrow$ ``T cells'' $\rightarrow$ ``Immune cells''. Similarly, applying $f_{\text{s}}$ to this element would return all other cell types in that belong to the same parent node ``T cells'', such as ``CD4+ T cells'' and ``NK cells''.

\subsubsection*{Hierarchical exntesnion of logistic regression}

Now that we have defined the hierarchical structure of the cell type annotation, we can extend the logistic regression model to incorporate this information. Similar to standard logistic regression, $\mathbf{x}_i$ as the $g$-dimensional vector of gene expression levels for cell $i$. Let $\mathbf{y}_i = [y_{i,1}, y_{i,2}, \ldots, y_{i,H}]$ be the set of labels for cell $i$, corresponding to each of the $H$ hierarchical layers. Similarly, the weight vector for layer $L_h$ is denoted as $\mathbf{w}_{L_h}$, where $h \in {1, 2, \ldots, H}$.

The probability of cell $i$ belonging to class $k$ at layer $L_h$ is now dependent on the hierarchical parents of this cell:

\begin{equation}
    p(y_{i,h} = k) = \begin{aligned}[t]
        & p(y_{i,h} = k | y_{i,h-1} = f_{\text{p}}(k)) \times \\
        & p(y_{i,h-1} = k | y_{i,h-2} = f_{\text{p}}(f_{\text{p}}(k))) \times \\
        & \cdots \\
        & p(y_{i,h_1} = f_{p}^{H-1}(k))
    \end{aligned}
\end{equation}

Notice that the conditional term like $p(y_{i,h} = k | y_{i,h-1} = f_{\text{p}}(k))$ is equivalent to the standard logistic regression model, restricted to the $f_s(k)$, i.e. the siblings of label $k$. 

Expanding the above equation using \cref{eq:logistic-regression-likelihood} yields the likelihood function for the hierarchical logistic regression model:

\begin{equation}
    p(y_{i,h} = k | \mathbf{x}_{i} \mathbf{w}) = \begin{aligned}[t]
        & \frac{\exp(\mathbf{w}_{L_h, k}^T \mathbf{x}_i)}{\sum_{j \in f_s(k)} \exp(\mathbf{w}_{L_h, j}^T \mathbf{x}_i)} \\
        & \frac{\exp(\mathbf{w}_{L_{h-1}, f_p(k)}^T \mathbf{x}_i)}{\sum_{j \in  f_s(f_p(k))} \exp(\mathbf{w}_{L_{h-1}, j}^T \mathbf{x}_i)} \\
        & \cdots \\
        & \frac{\exp(\mathbf{w}_{L_1, f_p^{H-1}(k)}^T \mathbf{x}_i)}{\sum_{j \in L_1} \exp(\mathbf{w}_{L_1, j}^T \mathbf{x}_i)}
    \end{aligned}
\end{equation}

The prior distribution for the weights $\mathbf{w}$ is the same Laplace distribution as in \cref{eq:logistic-regression-prior}. To address the issue of class imbalance, we now adjust the weights using the square root of the number of cells in each group. We employ square root scaling to moderate the penalty on larger groups, thereby maintaining a balanced emphasis across different classes.

This model is fit using \ac{ADVI}, and tha posterior distributions of the weights are used to asses the improtance of a gene to define a given class. 

\subsubsection*{Set of markers}

Running the model on the breast cancer atlas derived from \ac{scRNAseq} data, as described by \textcite{Wu2021-uq}, and subsetting the genes targeted by the two \ac{ISS} panels yielded a marker set with limited resolution. Specifically, the markers could distinguish only two hierarchical levels. At the first level, adequate markers allowed for differentiation among broad cell types: ``Immune", ``Epithelial'', and ``Stromal''. At the second level, the ``Immune" cells resolved further into ``B-cells", ``Myeloid", and ``T-cells". The ``Stromal" category subdivided into a mixed population of "Fibroblasts + PVL" and "Endothelial" cells. The analysis did not provide further resolution due to a poor choice of markers during panel design.

\subsection{Cell type assignmment in sparse signal setting}

We overlaid the ISS data with nuclei segmentation masks and allocated signals to the closest nuclei. We applied a conservative distance cut-off of 5$\mu$m — approximately double the nuclear radii — to minimise the likelihood of misannotation, resulting in approximately 30\% signal loss.

Cell type classification occurred in two steps. In the first iteration, nuclei possessing any markers corresponding to the broad categories (``Epithelial", ``Immune", ``Stromal") were assigned accordingly. In the subsequent iteration, nuclei initially classified as ``immune" or ``stromal" underwent further categorisation based on the presence of specific markers. Nuclei lacking proximal markers or exhibiting conflicting assignments received an ``unknown" classification.

\subsection{\acs{GLMM} for Multiregional Quantitative Analysis}

The primary objective is to elucidate the phenotypic, transcriptional, and compositional differences between clones. Given that clones generally do not intermingle, we restricted our analysis to highly clonal regions based on specific \ac{CCF} thresholds: CCF\textsubscript{clone} > 0.7 for P1, and CCF\textsubscript{clone} > 0.15 and 0.05 for P2-TN1 and P2-LN1, respectively. The lower thresholds for P2-TN1 and P2-LN1 account for the high levels of non-epithelial cells present in these samples.

Consequently, the dataset comprises regions categorised by dominant clone, with associated count or categorical data.

Standard statistical tests like Fisher's exact test easily handle categorical data, such as histological annotations. However, these methods are less applicable for count data, which are subject to variations in cell numbers and likely overdispersion (e.g. region-specific expression variation). To address these challenges, I employed custom \acfp{GLMM} designed to account for variable cell numbers and data overdispersion.

\subsubsection*{Intrinsic expression in specific cell types}

To quantify differential expression associated with clonal regions for a specific cell type, we record \ac{ISS} signals corresponding to nuclei classified under that cell type. I We model the distribution of observed \ac{ISS} expression signals $Y_{rg}$ for genes $g$ in regions $r$ using a clonal-specific expression rate $\beta_{cg}$,

\begin{equation}
    \beta_{cg} \sim \text{Uniform}(-10,10)
\end{equation}

Acknowledging the possibility of region-specific variation in expression, let's introduce a random effect variable $\alpha_{rg}$. This variable has a clonal-specific hierarchical prior $\sigma_{cg}$. The binary matrix $A_{rc}$ maps dominant clones to their corresponding regions.

\begin{align}
    \sigma_{cg} &\sim \text{HalfNormal}(0.05) \\
    \alpha_{rg} &\sim \text{Normal}(0, A_{rc}\sigma_{cg})
\end{align}

After adjusting for the number of nuclei of the relevant cell type 
$x_r$, the likelihood becomes:

\begin{equation}
    Y_{rg} \sim \text{Poisson}\left(x_r e^{A_{rc}\beta_{cg} + \alpha_{rg}}\right)
\end{equation}
    
This formulation accommodates both clone- and region-specific variations in gene expression, providing a more robust statistical model for analysing cell-type specific differential expression.

\subsubsection*{Cell-type composition}

To assess differential cellular composition between regions associated with clones, I computed the counts of nuclei categorised as specific cell types. I then modelled the distribution of nuclei counts $Y_{rt}$ attributed to cell type $t$ across regions $r$ as a mixed model. This model incorporates both a clone-specific frequency $\beta_{ct}$ and a region-specific random effect $\alpha_{rt}$, akin to the approach used for \textbf{cell-type specific expression}:

\begin{align}
\beta_{ct} &\sim \text{Uniform}(-10, 10) \\
\sigma_{ct} &\sim \text{HalfNormal}(0.05) \\
\alpha_{rt} &\sim \text{Normal}(0, A\sigma_{ct})
\end{align}

Given the sum-to-one constraint on cell type frequencies, I employed a softmax (logit) link function along with a Multinomial likelihood for the cell types:

\begin{equation}
\lambda_{rt} = A\beta_{ct} + \alpha_{rt}
\end{equation}
\begin{equation}
Y_{rt} \sim \text{Multinomial}\left(n=x_r, p=\frac{e^{\lambda_{rt}}}{1 + \sum_{i=1}^{T-1} e^{\lambda_{rt}}}\right)
\end{equation}

By incorporating these formulations, I ensure that the model accounts for the inherent constraints and random effects in differential cellular composition across regions.

\subsubsection*{IHC staining}
We counted the number of IHC-positive and IHC-negative nuclei in each region using QuPath. To enable the application of the aforementioned \textbf{cell-type composition} model, I treated IHC\textsuperscript{+} and IHC\textsuperscript{-} nuclei as two distinct cell types.

\subsubsection*{Inference and statistical testing}

I implemented the models using the \ac{NumPyro} library and employed \acf{HMC} for parameter inference \pcref{sec:bayesian-intro}. To identify quantiatively differences, I performed pairwise comparisons of clone-specific expression rates $\beta_{cg}$, for each pairs of clones.

\ac{HMC} generates empirical posterior distributions with a finite number of samples, limiting our ability to resolve small differences. Specifically, the quantiles are restricted to values of $1/n$, where $n$ is the sample size from the \ac{HMC} run. This limitation becomes particularly challenging when conducting multiple comparisons, as it necessitates the multiple testing correction. 

To enable the calculations of more extreme quantiles, I approximated the posterior distribution using the square of the mean divided by the standard deviation, or $\left(\frac{\mu(\Delta)}{\sigma(\Delta)}\right)^2$. This approximation follows a $\chi_{1}^{2}$. For differential composition and protein expression, I used a similar approach with the softmax-transformed $\beta_{ct}$ values. 

Although it diverges slightly from standard Bayesian principles, I treated the posterior distribution as analogous to a frequentist test statistic. I then applied multiple testing corrections to each comparison to maintain a \acf{FDR} below 0.1.

To maximise statistical power in clone-specific differential expression analysis, we excluded genes with a clone-agnostic average number of detected signals per region less than 0.3.

\subsection{Multimodal data visualisation}

The challenge of visualising spatial multimodal data, particularly in the project that combines \ac{BaSISS}, \ac{ISS}, and histological data, requires a tailored approach. Current methods such as \ac{TissUUmaps}, \ac{Omero}, \ac{WebAtlas}, and \ac{Napari} fall short in meeting specific visualisation requirements, especially for clone fields inferred from \ac{BaSISS} data.

To address this challenge, we created a specialised web-based tool, available at \href{https://www.cancerclonemaps.org/}{cancerclonemaps.org}. Our tool utilises a Flask backend and incorporates frontend technologies like `D3' for visual elements, `pixi.js' for efficient spot visualisation, `Leaflet' for tissue mapping, `Geoman' for map selection tools, and `Plotly.js' for the diagrams. This stack of technologies enables interactive exploration of our complex data sets, offering a unified platform to investigate histogenomic relationships. Consequently, this tool streamlines exploratory analysis and assists in identifying qualitative patterns across different data types.

\section{Discussion}

better technologies and better 