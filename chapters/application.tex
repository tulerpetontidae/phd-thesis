\chapter{Applications}
\label{sec:chapter-basiss-applications}

\section{Breast Cancer}

In 2020, breast cancer surpassed lung cancer to become the most diagnosed cancer type in the world \parencite{Sung2021-xv}. It accounted for approximately 11.7\% of all diagnosed cancer cases. This trend occurred irrespective of a country's development index, predominantly targeting the female population. Within this demographic, breast cancer constituted nearly a quarter of all diagnoses, and 15.5\% of all cancer-related deaths. An estimated 2.2 million cases were registered in 2020, and this number is expected to grow, partly due to the ageing population and the fact that females constitute the majority of the older population. Even steady improvements in treatment and early detection are unlikely to reverse this trend, as the deceleration in the decline of female breast cancer death rates appears to be aligned with a consistent increase in incident rates \parencite{Cronin2022-mc}.

Breast cancer typically originates within the ductal system of the breast, more specifically from a structure known as the \ac{TDLU}.\footnote{Surprisingly most papers name it as \textbf{the exclusive site of origin for all breast cancers} which seems to be unlikely since any dividing cell should have a capacity to cause cancer. Even Dana-Farber Cancer Institute explanation on breast cancer meantions it at their \href{https://www.dana-farber.org/health-library/articles/what-is-lobular-breast-cancer-}{website} (August 2023). Although the study by \textcite{Tabar2014-ea} suggest that ~ 25\% of cases actually originate in the duct and show no link to \ac{TDLU}.} The \ac{TDLU}, comprised of a terminal duct and multiple acini or lobules, functions as the site for milk production and secretion during lactation. When precancerous cells begin to actively divide (a process known as hyperplasia), they can spread throughout the ductal system. This fills the lumen of the ducts and lobules, a stage referred to as carcinoma in situ. Eventually, these cells may break through the basement membrane, invading the surrounding tissue in what is known as invasive carcinoma. At this stage, the cancer cells have the potential to enter the bloodstream and metastasise to other parts of the body.

The histological and molecular characteristic of breast cancer posess remarkable heterogeneity and have direct implications on patient survival and assigned treatment. Histologically, the main division is based on the appearance of cancer cells at the invasive stage to the ductal vs lobular normal tissue. The cancers are called 'lobular' or the 'ductal' even though it is improtant to note that this classification has nothing to do with the place of origin \parencite{Tabar2022-iq}.

A widely adapted breast tumour classification sheme devides it into 

The ductal epithelium consists of two histologically and transcriptionally distint layers - the luminal epithelial cells and basal myoepithelial cells. It is unclear which of these two cell types is the cell of origin for breast cancer, probably both could be. The wildy adapted 

In contemporary clinial practice 

genetic predisposition and known drivers

subclonal divercity and evolution

metastiasis

\section{Two cases of multifocal breast cancer}

The cohort includes eight tissue blocks from two patients (P1 and P2) who underwent a surgical mastectomy for a multifocal breast cancer. These patients were selected to permit a comparison between genetic and histological progression models in early breast cancer development29 (Fig. 1b,c). P1 had two separate oestrogen receptor (ER)-positive, human epidermal growth factor receptor 2 (HER2)-negative primary invasive breast cancers (PBCs) within a 5-cm bed of DCIS; we used tissue blocks from both PBCs (samples P1-ER1 and P1-ER2) and three regions from DCIS (samples P1-D1, P1-D2 and P1-D3). P2 had two separate PBCs of the ‘triple-negative’ subtype (lacking the ER, progesterone receptor and HER2). We sampled both PBCs (samples P2-TN1 and P2-TN2) and an axillary lymph node that contained metastatic cancer deposits (sample P2-LN1) (Fig. 1b).

\section{Charting histogenomic relationships}

Histology-driven sampling of well-defined stages of cancer progression can uncover mechanisms and markers of disease progression10,19,29,34. Up to two-thirds of PBCs contain both invasive cancer and intermixed DCIS, a non-obligate precursor lesion. How these distinct ‘stages’ of cancer development might relate to genetic diversification within the same tissue is generally unknown35 (Fig. 1c). To demonstrate that BaSISS can chart these relationships across entire tissue sections, we examined three PBC samples with intermixed invasive and DCIS histology: P1-ER1, P1-ER2 and P2-TN1 (Fig. 3 and Extended Data Figs. 5 and 6a–c).

BaSISS detected 2–4 subclones per PBC in accordance with bulk WGS data. Clone maps (Fig. 3a,e) and the quantitative clonal composition of 73 individually annotated microregions (Fig. 3b,f and Extended Data Figs. 5a,b and 6a,b) revealed that individual subclones form spatial patterns that were, by varying degrees, related to the histological progression states. Normal tissue elements, including immune aggregates and histologically normal ducts, appear unstained consistent with a wild-type status for the targeted clones (green and yellow contours, respectively; Fig. 3a,e). In P1-ER2, an area of hyperplasia was predicted and confirmed by LCM–WGS to be genetically unrelated to the cancer (blue contour; Figs. 2c and 3a).

In each PBC, the genetic and histological progression models were broadly consistent, in which the invasive disease was mainly composed of cells from the most recently diverged subclone: P1-red, P1-purple and P2-purple in samples P1-ER1, P1-ER2 and P2-TN1, respectively (Fig. 3b,f). By contrast, earlier diverging clones colocalized entirely or in part to the histological pre-invasive lesion: DCIS. For example, in P1-ER2, BaSISS predicted that green branch mutations were completely absent from the invasive compartment, a conclusion that is supported by three separate microdissections (LCM–WGS) from distant regions of invasive cancer in P1-ER2 (Fig. 2c and Extended Data Fig. 5c).

However, in each PBC, there was a subclone that spanned both DCIS and invasive histology, revealing that disconnects between histological and genetic progression states can exist. This was the case for clone P1-red in P1-ER1 and clone P1-purple in P1-ER2. These DCIS-invasive spanning clones could be distinguished from each other by hundreds of private mutations, including different inactivating driver mutations in PTEN, indicating parallel evolution along these divergent lineages that resulted in two distinct instances of cancer invasion (total mutation numbers label the phylogenetic tree branches; Fig. 3b). The spatial predictions of the BaSISS model of intraductal acquisition of PTEN mutations and PTEN protein loss was confirmed by LCM–WGS and IHC, respectively (Fig. 2c and Extended Data Fig. 5d). In sample P2-TN1, the only predicted driver point mutation was a deleterious mutation in the tumour suppressor gene TP53, and this was detected in both DCIS and invasive compartments and was also present in all cancer regions of the second PBC, P2-TN2, consistent with an early onset in the development of this cancer (phylogenetic tree; Fig. 3e,f). These data therefore suggest that many, if not all, of the genetic events necessary to initiate the invasive transition in these three cancers were acquired within the ducts, and subsequently both intraductal expansion and stromal invasion ensued.

\section{Phenotypic changes accompany progression}

Next, by integrating additional layers of spatial data, we sought to establish how phenotypic changes relate to genetic-state and histological-state transitions. In P1-ER1 and P1-ER2, consistent with a more proliferative phenotype, PTEN-mutant clone regions exhibited denser Ki-67 IHC nuclear staining, than PTEN wild-type ancestral clone regions (false-discovery rate (FDR) = 0.004 P1-red versus P1-orange; and FDR = 0.03 P1-purple versus P1-green) (Fig. 3c,d and Extended Data Fig. 5e). However, for a given genetic clone, the Ki-67 score was similar irrespective of whether it occupied a DCIS or invasive state, indicating that upregulation of Ki-67 is temporally related to acquisition of a PTEN mutation and precedes invasion.

By contrast, cellular resolution spatial transcriptomics analysis of P1-ER2 revealed that epithelial cell expression of several genes—CLDN4 (encoding claudin 4), ACTB (encoding β-actin), KRT5 (encoding keratin 5) and CTSL2 (encoding lysosomal cysteine protease cathepsin V)—differed between DCIS and invasive compartments occupied by the same, P1-purple, clone (Extended Data Fig. 5f). These transcriptional changes might therefore be considered more closely linked to the histological transition rather than genetic changes traced by this approach. Expression of CLDN4 was consistently lower in the invasive compartment than to each DCIS clone. However, for some genes such as ACTB, expression patterns changed in opposing directions in the invasive cancer relative to the sampled DCIS clone (expression is higher than P1-green DCIS (FDR = 0.02) and lower than P1-purple DCIS (FDR = 0.013)) or were highly specific to a genetically more distant DCIS clone (Extended Data Fig. 5f).

Attempts to isolate the changes associated with invasive transition might also be confounded by heterogeneity within the invasive compartment. In P2-TN1, we therefore sought to examine whether the two genetically distinct invasive subclones (P2-blue and P2-purple) were phenotypically distinct. The two cancer clones exhibited distinct morphological (nuclear and architectural) features (P = 0.04, Fisher’s exact test) (H\&E image insets; Fig. 3e,f) and occupied neighbourhoods with different stroma (FDR = 0.02) and immune cells such as myeloid cell densities (FDR = 0.08) (mini-image insets; Fig. 3e and Extended Data Fig. 6a–c). Transcriptional programs were also distinct, with statistically significant differences in gene expression for 12 of 91 genes between clones (Extended Data Fig. 6d). Together, these data indicate that the particular clones sampled can have a profound effect on attempts to identify the phenotypic changes implicated in driving or arising during histological progression.

\section{Growth patterns of pre-invasive clones}

To demonstrate that BaSISS can be used to chart growth patterns in relation to complex tissue structures, we turned our attention to three DCIS samples from P1 that spanned a tissue surface area of 224 mm2 (P1-D1, P1-D2 and P1-D3) (Fig. 4a and Extended Data Fig. 7a). The adult female breast comprises multiple, branching ductal systems, termed lobes, that extend from the nipple surface to the acini of the lobules, as illustrated in Fig. 4c36,37. DCIS arises from the duct epithelium and is considered a lobar disease as it typically involves the ducts and lobules of a single lobe38. Although DCIS is known to be genetically heterogeneous19, how DCIS clones are organized and grow through the wider duct system remains elusive39.

The clone maps generated for the three samples formed striking mosaics of mainly green and orange, and occasional blue and grey that localized to areas of histologically confirmed DCIS (Fig. 4a and Extended Data Fig. 7a). Immune clusters and occasional normal or hyperplastic ducts appeared white (unstained), consistent with a different genetic ancestry. In P1-D3, a 3-mm length of a large duct exhibited both a genetic and a histological transition from normal ductal epithelium to DCIS along its length, confirming that, although neoplastic involvement was extensive in this lobe, it was incomplete (Extended Data Fig. 7a). On dividing the glandular tissue into lobules (white dashed contours; Fig. 4a), it was apparent that a handful of lobules contained a single clone, but often multiple clones co-occurred. Indeed, we were surprised to observe that the same clones repeatedly co-existed within lobules that spanned centimetres of tissue. These appearances seem at odds with the traditional model of clonal competition in which a fitter clone generates localized monoclonal sweeps (Fig. 4c).

However, at finer, sublobular resolution, complete or near-complete clonal sweeps are the dominant pattern, as exemplified by assaying 146 representative microscopic regions that represent individual or small clusters of intimately related acini and ducts (beige contours; Fig. 4a). The existence of frequent clonal sweeps as inferred by BaSISS (Fig. 4b) was corroborated by LCM–WGS of additional microregions (Extended Data Fig. 7b). In some instances, including P1-D1-88 (Extended Data Fig. 7c) and P1-D2-0 (Fig. 4a,b,d and Extended Data Fig. 7d–f), clonal interfaces are directly observed within a continuous anatomical space. However, more commonly, rapid clone field transitions (see interactive maps (https://www.cancerclonemaps.org/)) coincided with the myoepithelial cell layer and/or basement membrane that define an acinus or ductule border. It thus transpires that the microanatomical structure of resident tissues can have, an as yet poorly understood, role in shaping observed subclonal architectures (Fig. 4a,c).

\section{DCIS clone-specific phenotypes}

Integration of histological and spatial gene expression data from serial sections revealed that the DCIS clones, P1-green and P1-orange, exhibit many phenotypic differences that are consistent across large tissue areas (Fig. 4d,e and Extended Data Figs. 7e,f and 8a,b). Histogenetic associations were very strong, with regions dominated by P1-green being more likely to have an intermediate rather than a low nuclear grade (P < 0.0001; Fisher’s exact test after Bonferroni correction), exhibit more nuclear pleomorphism (P < 0.0001), necrosis (P < 0.0001), vacuoles (P < 0.0001) and a non-solid architectural growth pattern (P < 0.0001) (Fig. 4d,e and Extended Data Fig. 7e,f).

Clone and cell type-resolved spatial gene expression analysis using targeted ISS further corroborated phenotype–genotype correlations. A total of 28 of 91 interrogated genes were differentially expressed by the two main clones (FDR < 0.1, fold change > 1.5 both ways; Extended Data Fig. 8a,b). Consistent with a higher nuclear grade, P1-orange epithelial cells exhibited higher expression of the cell-cycle regulatory oncogenes CCND1 and CCNB1 and the oncogene ZNF703, which have been linked to adverse clinical outcome40. Overall, architectural and nuclear appearances and gene expression profiles were remarkably lineage-specific, and it was particularly notable that these different patterns could also be appreciated spatially, in regions with sublobular, microscopic clone intermixing, adding weight to the clone composition predictions by the model (Extended Data Fig. 7d).

\section{Metastatic clones in a lymph node}

Lymph node metastasis is associated with higher rates of cancer mortality41. Whether it has an active role in facilitating cancer progression or simply reflects a more aggressive or distinct biology of certain clones is largely unknown. A substantial challenge is low cancer purity of diffusely infiltrated lymph nodes, which can make it difficult to separate cancer from immune cell-derived molecular signals. To demonstrate that BaSISS can facilitate the simultaneous study of cancer and immune compartments in such challenging cases, we analysed BaSISS, histological annotation and ISS targeted gene expression datasets from sample P2-LN1 (Fig. 5 and Extended Data Fig. 9).

BaSISS in P2-LN1 targeted 13 trunk and branch alleles, including point mutations and an expressed novel internal fusion in the CACNB1 gene that was co-amplified with the clinically targetable breast cancer oncogene HER2 in a breakage fusion bridge event (Fig. 5b and Supplementary Data Table 1). The model detected two clones (P2-blue and P2-orange) that formed spatially segregated patterns in P2-LN1 (Fig. 5a,d). Only P2-blue was detected in primary breast tumours (P2-TN1 and P2-TN2) (Fig. 3e and Extended Data Fig. 6b).

Detailed histological annotation, blinded to the clone territories, was performed using a combination of H\&E, CD45 and pan-cytokeratin IHC and identified multiple metastatic cancer growth patterns (coloured contours; Fig. 5a,c,d and Supplementary Table 2). Intersecting the clone maps and histological annotations revealed strong associations between the two detected clones and the two main histological growth patterns (P < 0.0001, Fisher’s exact test) (Fig. 5d). The P2-orange clone formed monotonous sheets of cancer cells, exhibited weak immunoreactivity for pan-cytokeratin and often occupied sinusoidal structures. By contrast, P2-blue cells stained more strongly for pan-cytokeratin and, when clustered, surround densely packed lymphocyte cores (Fig. 5c,d and Extended Data Fig. 9a–d).

We sought to determine whether transcriptional differences support the spatial inference of clones. Consistent with the known HER2 amplification, P2-orange expressed higher levels of HER2 (Fig. 5f and Extended Data Fig. 9c). A total of 17 of 91 genes were differentially expressed and many of these are implicated in critical biological cancer pathways and/or have recognized prognostic value, including CTSL2, VEGFA (encoding vascular endothelial growth factor receptor A) and CD24 (refs. 42,43) (Fig. 5f). Spatially plotting these genes confirmed that clone-specific expression patterns are recapitulated within multiple, spatially distinct expansions across more than 1 cm2 of tissue (Extended Data Fig. 9a–c).

Integration of spatial transcriptomics data also revealed that metastatic subclones occupied distinct immune microenvironments. Relative to P2-orange cells, P2-blue cells resided in neighbourhoods enriched for T cells and B cells (Fig. 5e,g). In fact, P2-blue cells frequently formed clusters around B cell-rich germinal-like centres, highlighting a potential clone-specific interaction with the adaptive immune system (Fig. 5c and Extended Data Fig. 9a,d). By contrast, P2-orange regions frequently resided inside the lymph node sinuses that were lined by endothelial cells expressing CD34 and PDGFRB (Fig. 5c and Extended Data Fig. 9f). Most of the immune cells in P2-orange regions were myeloid cells with expression profiles consistent with the existence of both M1 and M2 macrophages (CD163, CD68, HAVCR2 and FCGR3A), and the most highly enriched gene, CXCL8, is released by hypoxic macrophages44 (Fig. 5e). Indeed, relative to P2-blue, it emerges that P2-orange experienced more hypoxic conditions manifesting as higher cancer cell expression of VEGFA and necrotic regions (Extended Data Fig. 9e,f). Hypoxia signatures are associated with adverse clinical outcomes, probably because they reflect the emergence of environments that can select for hypoxia-tolerant clones and/or cancer proliferation rates outstrip neoangiogenesis45. Together, these data demonstrate how BaSISS clone maps allow one to spatially relate such variation in microenvironments to individual clones.

\section{Discussion}

Here we present BaSISS, a pipeline that combines a highly multiplexed fluorescence microscopy-based protocol and algorithms to map and phenotypically characterize the unique set of subclones of cancer. These maps served as the basis for further spatially and single-cell-resolved molecular and histological characterization of each clone. Applying BaSISS to a series of samples from the key stages of breast cancer progression—carcinoma in situ, invasive cancer and lymph node metastasis—it is notable that virtually every sample exhibited a spatial organization of clones, which warrants further investigation in larger cohorts. The fact that nearly all clones examined in this dataset displayed distinct clone-specific gene expression, stromal and immune microenvironments and microanatomical niches highlights the functional relevance of at least some subclonal diversification.

The ability to chart clonal growth patterns and clone-specific genetic underpinnings of the tumour microenvironment is likely to be instrumental in elucidating how different evolutionary processes operate and manifest across different cancer types—or even in histologically normal tissues46. Understanding the forces of malignant progression, especially invasion and metastasis, and how interactions with the tumour microenvironment shape clinical outcomes10 appear of particular importance. Detailing the functional and microenvironmental characteristics of different clones is also relevant as some part of subclonal diversity in tumours may be due to selectively neutral drift, but the exact extent remains debated.

Particular advantages of the technology are that it is capable of interrogating very large tissue sections on the scale of squared centimetres, which enables studying entire cross-sections of smaller tumours. It is also comparably cheap, unlike solely relying on sequencing-based methods47. The three main limitations of the approach are relatively low sensitivity, which currently precludes single-cell genotyping, a reliance on RNA with the resulting variation in gene expression levels of targeted transcripts, and the fact that clone-defining mutations need to be detected first by separate sequencing-based assays. Greater sensitivity and spatial resolution may be achieved by including more targets per clone and by favouring mutations with higher predicted expression levels, for example, in higher copy number states. A switch to hybridization-based sequencing and direct RNA-binding probes may also improve base-specific detection by several fold48,49. Further discussion of the implications of our observations and limitations of the method is provided in a Supplementary Note.

It is often stated that “nothing in biology makes sense except in the light of evolution”50, which is likely to be true for cancer biology. The ability to spatially locate and molecularly characterize different cancer subclones adds essential features to the spatial-omics toolkit. It provides a robust evolutionary framework that is necessary to interpret the biological relevance of many of the more plastic spatial characteristics of a cancer. Future widespread applications of spatial genomics approaches such as BaSISS will uncover how cancers grow in different tissues and allow us to track, trace and characterize the ill-fated clones that are responsible for adverse clinical outcomes.

