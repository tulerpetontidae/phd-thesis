\chapter{Outlook}
\label{sec:chapter-discussion}

\section{Summary}
Cancer is fundamentally a spatial disease as its origin and progression are linked to specific anatomical locations and microenvironments within the body. These spatial relationships between cancer cells and adjacent tissues offer key insights into tumour onset, evolution, and treatment response \pcref{sec:chapter-introduction}. Despite this importance, current technologies constrain our understanding of the interactions between subclones and the microenvironment. Existing spatial technologies targeting the genome either suffer from limited spatial resolution or employ indirect methods based on a limited subset of genomic changes, thereby complicating phylogenetic inference.

In this thesis, I developed the key computational components for one of the first method to chart cancer evolution in tissue space \pcref{sec:chapter-basiss-model}. 
I designed algorithms that integrate clonal maps with various types of auxiliary data, including \ac{ISS}, \ac{IHC}, and histology, to provide a comprehensive characterisation of clones and the tumour microenvironment \pcref{sec:chapter-basiss-multimodal}

Applying these methodologies to multifocal breast cancers, I charted the spatial organisation of clones through three stages of cancer progression: \acf{DCIS}, invasive carcinoma, and lymph node metastasis \pcref{sec:chapter-basiss-applications}. This study revealed a remarkable degree of segregation among histologically distinct cancer clones within the breast's ductal system. It tracked the phenotypical changes accompanying progression from \ac{CIS} to the invasive stage and confirmed that genetically indistinguishable clones can exist in both invasive and \textit{in situ} states. The study also showed how cancer clones adapt to unique ecological niches within the lymph node, suggesting possible interactions with the immune system. Across all three stages examined, the territories occupied by different clones exhibited distinct transcriptional and histological features, as well as diverse microenvironments.

These results demonstrate the utility of spatial genomics to elucidate the mechanisms that govern cancer evolution and spatial ecology.

\section{Current limitations and future solutions}

The approach comprises two distinct components: lineage tracing with \ac{BaSISS} and the characterisation of cell phenotype and microenvironment. As these components operate independently and possess their own specific limitations and potential solutions, I will discuss them separately.

\subsection{Spatial lineage tracing with \acs{BaSISS}}

While the \ac{BaSISS} protocol offers the unique ability to map cancer clones across square centimetres of tissue, there are several limitations that are important to consider. Some of these are intrinsic to the technology, such as its emphasis on mutations in the coding region and its dependency on prior mutation detection and phylogenetic reconstruction. Other limitations, such as the lack of detailed phylogeny, low signal yield, and standardisation issues, could possibly be addressed with existing technology. Overcoming these constraints would improve both the resolution and the granularity of cancer evolution characterisation. 

\subsubsection*{Prior knowledge of genome alterations}

\ac{BaSISS} is a targeted method that require both prior knowledge of the mutations and requires a panel redesign for every individual cancer case, drastically increasing the cost of new experiments. This contrasts with spatial sequencing methods \parencite{Zhao2022-xd,Erickson2022-zh} which do not require prior information and infer clone maps based on copy number profiles. While it is possible to adapt \ac{BaSISS} to this strategy by designing probes for a set of expressed genes distributed across the genome, such an approach would only yield copy number profiles. Consequently, this would forfeit \ac{BaSISS}'s primary advantage: the ability to detect point mutations.

\subsubsection*{Prior phylogeny reconstruction}
Initial detection of subclones as well as tree reconstruction was done based on multi-regional sequencing data. This method is often not sensitive enough to resolve clones in cases where the \ac{CCF} is low. An alternative approach to improve point mutation detection and tree resolution could involve techniques like single-cell \ac{WGS}. Currently, \ac{LCM} might offer an even better option, owing to its superior ability to detect point mutations.

When employing \ac{LCM} one could slightly modify the research workflow. Instead of first inferring cancer clones and subsequently exploring their association with histology, histological features could guide the selection of \ac{LCM} regions for genotyping. While this strategy introduces some bias, it could more effectively elucidate the relationship between histology and genomics. \ac{BaSISS} could then extend genotyping capabilities to regions not characterised by \ac{LCM}. For an illustration of the additional phylogenetic details that can be revealed through such methods, refer to \cref{fig:applications-evolution-tree}.

\subsubsection*{Subclone resolution}
The number of probes defining clonal expansion is critical for accurate subclone mapping. For instance, the P1-green clone \pcref{fig:applications-maps-DCIS} was defined solely by a point mutation in the \gene{KIAA0652} gene. Fortunately, this gene was highly expressed, mitigating the risk of failing to resolve this particular clone. Although gene expression levels were taken into account during probe design, predicting gene expression patterns is challenging. This is particularly true when expression varies non-linearly with changes in copy number (see \cref{sec:basiss-model-assumptions} for details). Additionally, it is difficult to forecast the efficacy of padlock probes, such as their specificity. Therefore, it is crucial to include as many genomic alterations as possible in the probe design.

% \explain{}{explain this with figure}
A fundamental limitation of the \ac{BaSISS} approach, and any other method targeting RNA, is the inability to trace clones not defined by mutations in the coding region. The mutation burden can vary considerably across different cancer cases and types, with the number of coding mutations per tumour ranging from 1 to 10,000 \parencite{Martincorena2015-br}. As a result, some tumours are more compatible with \ac{BaSISS} analysis. In particular, hypermutant tumours, which contain a significantly higher number of mutations compared to typical cases, may be especially amenable to investigation, given the greater likelihood that their clones will accumulate mutations in the coding region.

When the task involves tracing hundreds of mutations, accomplishing this with \ac{BaSISS} is relatively straightforward. The 56 probes that were used to map the P1 clone are not a technological limitation for the \ac{ISS} protocol. Modern \ac{ISS} experiments can resolve up to 300 genes, which, in the context of \ac{BaSISS}, equates to the detection of 150 allelic variants plus an additional 150 reference alleles.

\subsubsection*{Signal yield}

A significant drawback of the \ac{BaSISS} protocol is its low yield of signals, which arises from two main factors: inefficient sequencing-by-ligation chemistry and limited targetable nucleic acids. Unlike standard FISH technologies, where probes can target multiple regions of the same gene to enhance detection, \ac{BaSISS} can target \acp{SNV} and structural variants with exactly one probe.

However, there is room for improving the sequencing chemistry, as demonstrated in newer versions of \ac{ISS} protocols. For instance, hybridisation-based \ac{ISS} (HybISS) \parencite{Gyllborg2020-uq} and direct RNA-targeted \ac{ISS} (dRNA-HybISS) \parencite{Lee2022-ha} offer improvements. The former one exchanges sequencing by ligation with a more efficient hybridisation step. The latter bypasses the need for reverse transcription required for cDNA generation. Nevertheless, it is essential to consider that RNA ligases used in these improved methods may have different substrate specificities, which could impact ligation accuracy and the off-target effect.

\subsubsection*{Standardisation}

Standardisation of the \ac{BaSISS} protocol could substantially enhance its utility and reproducibility across different laboratories. The \ac{BaSISS} protocol involves a diverse array of equipment, reagents, and computational infrastructure, all of which must be meticulously operated. Additionally, trained personnel are required to execute each stage with minimal variability. Even within the scope of this thesis where the experiments were conducted by the same individuals, I observed considerable variability in signal detection between slides. Attempts to replicate the experiments in a different facility have highlighted the challenges associated with even minor issues, such as imperfect alignment of the glass slide under the microscope between reaction cycles. These seemingly trivial errors took months to identify and led to significant degradation in data quality. Transitioning the protocol to industrial-standard machines, similar to those used in the \ac{Xenium} technology, could make the method more reproducible and accessible to a broader research community.

\subsubsection*{Scope of future applications}

Is this technology a silver bullet for solving the challenges of spatial genomics? Probably not yet. Despite its many advantages, the \ac{BaSISS} approach has some inherent drawbacks. While it is cost-effective for profiling multiple samples from the same tumour, the need for customised panel design for each experiment to capture \acp{SNV} elevates the costs. It improves upon RNA sequencing methods which can only resolve clones based on copy number profiles; however, \ac{BaSISS} still loses to \ac{LCM}-\ac{WGS} due to its fundamental constraint: the necessity for clone-specific mutations in the coding region, which are limited. Additionally, despite the avenues for improvement, this approach currently suffers from low sensitivity and lacks single-cell level genomic characterisation. Consequently, spatial lineage tracing remains a significant challenge.

\explain{}{Despite the limitations, profiling the clonal structure of normal tissues remains feasible. Even though nomral cells generally exhibit a lower mutation burden, \ac{LCM} studies identify up to 100 mutations in their coding regions. Although, one need to keep in mind the age of donors, for example in \textcite{Li2021-th} they were 85 to 93 years old.}

However, I envisage that \ac{BaSISS} could serve as a valuable supplement to other spatial genomics tools like \ac{LCM} \pcref{box:technologies}, particularly for deeply investigating the evolutionary history of hypermutant cancers.

The model I developed (detailed in \cref{sec:chapter-basiss-model}) along with the accompanying infrastructure code, is designed to handle moderately noisy \ac{BaSISS} data in future experiments. This model is capable of mapping an arbitrary number of clones across centimetres of tissue, assuming each clone possesses at least one well-behaved, clone-specific mutation. While there is room for improvement, as discussed in \cref{sec:basiss-model-assumptions}, further ameliorations will be needed as experimental methods advance to offer single-cell-level resolution. Such advancements will invite substantial changes to the model's structure. Although the data generation process will share many similarities with the current approach, future data sets will likely not require binning or \ac{GP} to manage sparsity. I anticipate that future models will employ clustering approaches similar to Bayesor \parencite{Petukhov2022-pv} or pciSeq \parencite{Qian2020-mp}.

\subsection{Phenotype and microenvironment characterisation}

The auxiliary data acquired for the study presented in this thesis is largely suboptimal by current standards. This limitation arises from the fact that the experimental setup was originally designed in 2016\footnote{It is strange to think that the original experiments took place just as I began my undergraduate studies.} by \ac{lucy}, \ac{jessica}, \ac{mats}, and \ac{peter}, predating the rapid advancements in the field of spatial omics. Consequently, the workflow detailed in \cref{sec:chapter-basiss-multimodal} should be viewed primarily as a proof of concept and should be replaced with modern methodologies in future research.

\subsubsection*{Histological characterisation}

Histological features offer a straightforward method of assessment, requiring often only simple staining to reveal a wealth of information. A trained histopathologist can interpret cell arrangements and nuclear shapes, which often suffice to define the micro-anatomy of the tissue that impose structural constraints on the tumour's environment \pcref{sec:intro-tissue-organisation}. Despite its utility, histopathological annotation performed by human is labour-intensive, subjective and difficult to scale.

Digital histopathology offers an alternative approach. Besides feature extraction, it is possible to conduct semantic segmentation of the regions and extract its micro-anatomical structures \parencite{Kiemen2020-dc}. Recent studies employing artificial intelligence have demonstrated that the histological characteristics of tumour cells correlate with underlying genomic alterations \parencite{Fu2020-cp, Kather2020-bt}. However, a significant challenge lies in integrating slide-level bulk genomic data with tile-based digital features \parencite{Shmatko2022-to}. Spatial genomics can address this issue, provided that a sufficiently large dataset is generated.

\subsubsection*{Spatial proteomics and transcriptomics}

Spatial transcriptomic and proteomic technologies have undergone significant advancements in recent years \parencite{Lewis2021-ic, Mund2022-kf}. This progress has made the acquisition of single-cell level proteomics and transcriptomics increasingly routine \pcref{sec:mapping-the-cancer-ecosystem,box:technologies}. Although these technologies focus on different modalities -- for example, protein profiles more readily identify immune cells -- they yield rich data sets. These data are sufficient for mapping the spatial distribution of cell types at, or slightly above, the single-cell level.

Highly multiplexed antibody detection methods yield rich data sets that facilitate the identification of functionally recurrent neighbourhoods within cancer tissue \parencite{Danenberg2022-zb, Jackson2020-em, Nirmal2022-sq, Schurch2020-lp, Wang2023-bo}. Given that proteins serve as primary mediators of cell-cell interactions, these technologies also enable the direct observation of cell-cell contacts, providing unequivocal evidence of interactions \parencite{Nirmal2022-sq, Wang2023-bo}.

To date, spatial transcriptomics assays for cancer have primarily employed \ac{Visium}, a technology that enables whole-transcriptome profiling and the mapping of various niche cell types, guided by rich single-cell transcriptomics reference data \parencite{Andersson2021-pu, Stahl2016-nq, Wu2021-uq, Berglund2018-gh, Moncada2020-ck, Qi2022-by, Ji2020-gn, Ravi2022-ut, Wu2021-wb, Gouin2021-zx, Barkley2022-gx, Erickson2022-zh}. However, because these methods operate at a supercellular level, they are mostly limited to describe broad patterns of co-occurrence and offer only indirect evidence of functional interactions \pcref{fig:intro-cancer-neighbourhoods}. The recent commercialisation of fluorescent probe-based technologies, such as \ac{Xenium} and \ac{Merscope}, is likely to shift this landscape, allowing for transcriptional characterisation at the single-cell level.

Overall, the success of spatial transcriptomics and proteomics, relative to other spatial modalities, will likely shape the field's trajectory towards phenotypic characterisation of cell communities in the coming years.

\subsubsection*{Towards comprehensive spatial models of carcinogenesis}

The statistical models outlined in \pcref{sec:modalities-glmm} serve to delineate compositional and expression differences between preselected regions occupied by distinct clones. With the availability of richer single-cell level features, compositional differences can be clustered into recurrent neighbourhoods using methods such as topic modelling or graph-based models, further enriching the description of tumour microenvironment \parencite{Danenberg2022-zb, Jackson2020-em, Nirmal2022-sq, Schurch2020-lp, Wang2023-bo}.

\figuretextwidth{discussion-hypothetical-model.pdf}{discussion-hypothetical-model}{Conceptual comprehensive spatial model}{Multiple spatial modalities at the single-cell level offer rich data for analysing cancer cell biology. A graph-based framework provides an effective approach for integrating these data types. Inspired by \textcite{Fischer2023-go}}

However, richer multimodal single-cell data enable more refined statistical approaches for modelling interactions between intrinsic and extrinsic cellular components. Graph-based representations preserve spatial relationships and offer a framework for modelling intercellular communication, as demonstrated by the NCEM method \parencite{Fischer2023-go}. While fully mechanistic modelling of cellular components poses significant challenges, neural network-based approximations could partly represent cell behaviour. I anticipate that the integration of multi-modal data within a graph-based model, which explicitly considers cell phenotypes as functions of both their intrinsic characteristics and local neighbourhood, will advance our understanding of the roles of genome and microenvironment in cancer cell behaviour and cellular plasticity \pcref{fig:discussion-hypothetical-model}.

\section{Opportunities for studying breast cancer}

The study detailed in \cref{sec:chapter-basiss-applications} demonstrates a surprising degree of spatial organisation among cancer clones. This work highlights the efficacy of cancer lineage tracing technology in establishing links between evolutionary lineages, their phenotypes, and their microenvironments. This technology thus offers valuable insights into cancer cell plasticity from an evolutionary perspective. While most of the findings are discussed in \cref{sec:application-discussion}, this section aims to situate these results within the broader context of unresolved questions concerning breast cancer progression, closer to the one listed in \cref{sec:application-open-questions}.

\explain{}{\marginfig{side-plot-evolutionary-framework.pdf} Understanding the evolutionary underpinnings of phenotypic variability aids in making accurate comparisons to elucidate biological questions like ``What phenotypic changes accompany the invasion process?'' }

\subsubsection*{Mammary gland development and ductal system colonisation}

Understanding early evolution and clonal dynamics within the ductal system is crucial for grasping how breast cancer develops its malignant potential. Previous studies have reported extensive heterogeneity of \acl{CIS}, however they did not address their precise spatial distribution over the large tissue areas \parencite{Casasent2018-gx,Nishimura2023-mk,Yates2015-xk}. 

The observed patchwork pattern of microscopic clonal sweeps within the \ac{DCIS} is striking and prompts questions about its formation. Several explanations for these growth patterns exist, including clonal co-operation and genetic drift, as discussed in \cref{sec:application-discussion} \parencite{Janiszewska2019-zq,Turajlic2019-sr}.

This observed pattern also aligns with simulations conducted by \textcite{West2021-ar}, which predict that local clonal sweeps arise due to the spatial constraints inherent in the fine branches of the ductal system. However, this model considers the mammary gland as a static system, neglecting its dynamic nature and life-long remodelling \pcref{sec:applications-normal-breast-development}. Developmental processes within the mammary gland could play a role in the clone spread.

The first detected clonal expansion in this case roughly coincides with puberty, a period of active ductal system formation. This expansion could have been facilitated by such developmental processes, stimulating cancer cells to proliferate and disseminate throughout the newly formed ductal and lobular units. Subsequent expansions may also align with other periods of breast remodelling, such as pregnancy, although we lack data for the P1 case in this context.

Within the \ac{DCIS}, we observed a distinct frontline between cancer clones without many examples of intermingling. This observation is intriguing from a clonal dynamics standpoint ass it raises questions as to what extend a `fitter' clone may outcompete others colonising ductal system and why do we observe complete local sweeps. The human breast continues to develop acini with each menstrual cycle \parencite{Javed2013-ew}, providing additional space for proximal clones to expand. Once colonised, lobules may become inaccessible to other clones, a phenomenon observed in bacterial colonisation of porous landscapes \parencite{Conwill2022-sp}.

Determining the extent to which this pattern results from cancer spreading within a static system versus the role of natural breast development, is challenging based on a static 2D snapshot of a dynamic 3D structure. However, the \ac{BaSISS} approach can be applied to consecutively capture the three-dimensional structure of clonal distribution. Such data should clarify the degree of clonal mixing at a global level, thereby providing a foundation for future evolutionary simulations. Additional genomic studies of individual lobules, e.g. with \ac{LCM} may help dissect clonal dynamic further and understand local patterns of colonisation. 

\subsubsection*{Cancer plasticity and invasion}

Prior research suggests that genetically identical clones can coexist in both \acl{CIS} and invasive states \parencite{Casasent2018-gx}. Moreover, the microenvironment plays a role in initiating invasion \parencite{Sinha2021-mf,Risom2022-uw} \pcref{fig:applications-epithelial-plasticity}. Nonetheless, it is evident that not all clones possess the capacity for invasion, and specific genetic backgrounds may be required for this capability.

The invasive growth patterns discussed in \cref{sec:chapter-basiss-applications} support this model. Invasive clones occupy the ductal system, occasionally co-localising with non-invasive clones \pcref{fig:applications-maps-PBC-P1}. While the absence of direct experimental evidence makes it challenging to assert whether non-invasive clones lack invasive potential, statistical analysis across large cohorts could reveal correlations between the genetic backgrounds of invasive and non-invasive lineages. Such analysis could also quantify associated changes in cell phenotype.

It is plausible that \ac{CIS} clones correlate with specific microenvironment types, as their histological features differ significantly. Unfortunately, the cell-typing analysis was not sufficiently sensitive to detail the functional differences between the niches associated with invasive and non-invasive clones, such as those proposed for neutrophils \parencite{Sinha2021-mf} and desmoplastic stroma \parencite{Risom2022-uw}. Future studies, leveraging advancements in spatial transcriptomics and proteomics profiling, should elucidate these microenvironmental differences.

\subsubsection*{Lymph node metastasis and environment interactions}

While lymph nodes actively execute adaptive immune responses, their role in cancer progression remains ambiguous in part due to the challenges posed by the low cancer purity in the diffuse infiltration of these structures. Genomic studies have reported extensive heterogeneity among cancer cells within lymph nodes \parencite{Pal2021-rf,Barry2018-el,Bao2018-kj}, however they lacked spatial details to functionally link genetic heterogeneity to tumour microenvironment.

The clonal patterns discussed in \cref{sec:applications-LN} suggest ongoing evolution within the lymph nodes. However, given that lymph node metastases have been reported to originate from minor clones at primary sites \parencite{Bao2018-kj}, these patterns could result from polyseeding by under-sampled clones in the primary tumour.

A striking observation presented in \cref{sec:applications-LN} is the specialisation of clones in fostering specific ecosystems.\explain{}{\marginfig{side-plot-saturn-LN.pdf} The P2-blue clone occupies a unique niche around B-cell cluster, as illustrated in \cref{appfig:appendix-applications-LN.pdf}} One clone is mostly avoiding immune presence within the sinuses, while another consistently clusters around B-cell aggregates. This indicates clone-specific interactions with the adaptive immune response. To the best of my knowledge, the phenomenon of cancer clones encircling B-cell clusters has not been reported previously. These observations suggest that clone-immune interactions within the lymph node could influence immune cell behaviour at primary sites, either by attracting immune attention or by contributing to immune evasion.

Future spatial studies involving appropriately paired cohorts should further clarify the nature of cancer-immune interactions within both the lymph nodes and primary tumours.

\section{Concluding remarks}
\label{sec:discussion-colcluding-remarks}
This thesis addresses a gap in cancer genomics by introducing a method capable of tracing clones across large tissue sections at single-nucleotide resolution. The computational techniques developed here enable the integration of experimental \ac{BaSISS} data with multiple phenotypic layers, such as \acl{IHC}, \acl{ISS}, and histology. Together, these methods enabled the characterisation of individual breast cancer clones and their native microenvironments. 

The proposed approach allows for the simultaneous observation of evolving clones, their phenotypes, and the selective environment in which they reside. As such, the utility of this method extends beyond the study of breast cancer progression. At the example of some of CRUK's \href{https://cancergrandchallenges.org/}{Cancer Grand Challenges}, I will illustrate how this spatial genomics approach could be applied to address other fundamental questions of cancer research. Specifically, the approach could advance the research themes of `Tumour cell plasticity', `Normal phenotype', `Lethal vs non-lethal cancers' and `Cancer and ageing'.

\textbf{Cancer cell plasticity.} The emergence, spread, and treatment resistance of cancer are often attributed to the ability of cancer cells to alter their states - a phenomenon known as `cell plasticity'. The origins of this diversity remain unclear. Part of it likely stems from irreversible state changes caused by genetic alterations selected during clonal evolution. Other state changes may be reversible, induced by triggers within the microenvironment. Therefore, methods that enable simultaneous profiling of the genetic composition (intrinsic component), expressed phenotype (a measure of plasticity), and microenvironment (extrinsic component) are necessary to accurately model cell plasticity.

\textbf{Normal phenotype.} Although experiments in model systems suggest that a limited number of driver mutations can initiate malignancy, tissue studies show that cells with multiple driver mutations can still appear histologically normal \pcref{fig:applications-evolution-tree}. This points to the significant role of the microenvironment, in conjunction with genetics, in determining tumour phenotype. Notably, cells with identical genotypes can behave differently in varied microenvironments, and vice versa. Therefore, studies that profile genetic composition, microenvironment, and histological appearance are essential for understanding the mechanisms enabling cells to transition between normal and malignant states.

\textbf{Lethal vs non-lethal cancers.} While some cancer clones stay confined within their tissue of origin and thus pose a relatively low risk to the patient, other clones acquire the ability to invade surrounding tissues. A subset of these invasive forms may ultimately metastasise to distant organs, substantially elevating the risk of a fatal outcome. Current screening methods often fail to detect the emergence of these `lethal' clones, leading to frequent over-diagnosis and subsequent over-treatment. The specific combination of genetic alterations and microenvironmental factors that render a clone `lethal' remains elusive. Spatial omics techniques hold the potential to not only refine existing histology-based biomarkers but also to discover new biomarkers that could improve patient stratification.

\textbf{Cancer and ageing.} Ageing triggers a cascade of cellular and tissue-level changes. At the cellular level, it exacerbates genomic instability, leading to a higher mutation rate. At the tissue level, it weakens immune functions and remodels the extracellular matrix. These alterations render tissues increasingly vulnerable to abnormal cell proliferation. However the exact mechanisms of these varied processes remain elusive and spatial genomics and transcriptomics approaches appear well suited to measure how ageing microenvironments affect clonal evolution. Moreover, the processes of ageing may be highly tissue specific, therefore requiring detailed analyses across tissue types to better understand the role of ageing in cancer development.

In conclusion, the spatial genomics techniques developed in this thesis provide a unique opportunity for studying the evolution of cancer clones within their native tissue environment. As demonstrated, utilising these methods to map tumour evolution across various tissues holds great potential for addressing key research challenges in the field of cancer biology.